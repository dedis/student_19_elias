\section{Conclusions}
Among the alternatative gossip-based aggregation implementations in the first part of the project, the new mask protocol is the best performing, since it keeps the same reasonably short time to produce a cosignature, doesn't have extreme outliers in the protocol duration. and while the number of messages exchanged is higher, the amount of data transferred is lower than the previous implementation.
Taking into account the gossip protocol model change from a push-only to a push and pull one, the performance improvements on these metrics are reasonable.
With the mask implementation we have also experimentally backed up the property of gossip protocol theory where it grows roughly like a function $O(log(n))$ of the number of nodes $n$.

For the mask aggregation protocol, the results are not good because of the mentioned constant propagation of all possible combinations of multi-signatures, which creates many unnecessary messages being sent around. The model had good intentions of trying to find the multi-signatures that can be aggregated to reach the threshold much faster by pulling these combinations, but this created not only an increase in the number of messages and bandwidth used, but also the duration of the protocol. This is because not only single signatures are being propagated, but also all possible multi-signature combinations are being passed around between nodes. In Figure~\ref{fig1time}, we can observe that with more nodes, the duration increases almost in an exponential way, which is not good at all.

For the second part of the project, the hybrid protocol implemented using ONet has a better performance than the existing gossip protocol, since it creates trees in the first part of a propagation round, and then gossips the message in the second part of it. This allows the new protocol to send less messages if the tree part of the propagation doesn't fail. If failure in an internal node causes it to block their children in the tree part, this problem will be overcome in the gossip part of the propagation.

The newly implemented protocols still have some limitations.
They do not adapt to the properties of the network, such as the network message delay, because ideally, a gossip protocol would send fewer messages per second in a slower network.
Due to the inherent randomness in the gossip protocol and the unpredictability of network message delays, there is no definite upper limit for the protocol duration.
Membership needs to be closed for the duration of the protocol, and all nodes need to know each other's identities and public keys beforehand.
For security against man-in-the-middle attacks, the protocol needs authenticated channels for messages.

There are some improvements that could be future work to expand the possible uses and improve the speed and efficiency of the protocols. The selection of peers who will receive a rumor in a gossiping round could be done in a different way than random picks. There is also the pending completion of the implementation of the protocol that uses homomorphic subtraction of signatures to use in a different aggregation. This would allow earlier aggregation and in theory this would reduce the bandwidth used. Finally, formal proof of all of the protocol's properties, and a more complete statistical analysis would be useful to make these implementations more attractive to applications that could potentially use them.