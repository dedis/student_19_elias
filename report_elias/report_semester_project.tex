%%%%%%%%%%%%%%%%%%%%%%%%%%%%%%%%%%%%%%%%%%%%%%%%%%%%%%%%%%%
% EPFL report package, main thesis file
% Goal: provide formatting for theses and project reports
% Author: Mathias Payer <mathias.payer@epfl.ch>
%%%%%%%%%%%%%%%%%%%%%%%%%%%%%%%%%%%%%%%%%%%%%%%%%%%%%%%%%%%
\documentclass[a4paper,11pt,oneside]{report}
% Options: MScThesis, BScThesis, MScProject, BScProject
\usepackage[MScProject,lablogo]{EPFLreport}
\usepackage{xspace}

\title{ONet Implementation of Gossip-based Signature Aggregation}
\author{Elias Manuel Poroma Wiri}
\responsible{Prof. Bryan Ford}
\supervisor{Gaylor Bosson}
\othersupervisor{Cristina Basescu}
%\coadviser{Second Adviser}
%\expert{The External Reviewer}

\newcommand{\sysname}{FooSystem\xspace}

\begin{document}
\maketitle
%\makededication
%\makeacks

\begin{abstract}
The \sysname tool enables lateral decomposition of a multi-dimensional
flux compensator along the timing and space axes.

The abstract serves as an executive summary of your project.
Your abstract should cover at least the following topics, 1-2 sentences for
each: what area you are in, the problem you focus on, why existing work is
insufficient, what the high-level intuition of your work is, maybe a neat
design or implementation decision, and key results of your evaluation.

TENTATIVE TOTAL SIZE 21 pages without appendix
\end{abstract}

\maketoc

%%%%%%%%%%%%%%%%%%%%%%
\chapter{Introduction}
%%%%%%%%%%%%%%%%%%%%%%

The introduction is a longer writeup that gently eases the reader into your
thesis~\cite{dinesh20oakland}. Use the first paragraph to discuss the setting.
In the second paragraph you can introduce the main challenge that you see.
The third paragraph lists why related work is insufficient.
The fourth and fifth paragraphs discuss your approach and why it is needed.
The sixth paragraph will introduce your thesis statement. Think how you can
distill the essence of your thesis into a single sentence.
The seventh paragraph will highlight some of your results
The eights paragraph discusses your core contribution.

Tentative size: 1 page.

%%%%%%%%%%%%%%%%%%%%
\chapter{Background}
%%%%%%%%%%%%%%%%%%%%

The background section introduces the necessary background to understand your
work. This is not necessarily related work but technologies and dependencies
that must be resolved to understand your design and implementation.

Tentative size: 4 pages.

\section{Gossip protocols}
Lorem Ipsum

\section{Multi-signatures}
Lorem Ipsum

\section{Existing implementations}
Lorem Ipsum

\subsection{BLS CoSi}
Lorem Ipsum

\subsection{BLS CoSi with binary tree aggregation}
Lorem Ipsum

\section{The Cothority Overlay Network Library - ONet}
Lorem Ipsum


%%%%%%%%%%%%%%%%%%%%%%%%
\chapter{Design and Implementation}
%%%%%%%%%%%%%%%%%%%%%%%%

Introduce and discuss the design decisions that you made during this project.
Highlight why individual decisions are important and/or necessary. Discuss
how the design fits together.

The implementation covers some of the implementation details of your project.
This is not intended to be a low level description of every line of code that
you wrote but covers the implementation aspects of the projects.

Tentative size: 6 pages.

\section{Gossip protocols}
Lorem Ipsum

\subsection{Mask protocol}
Lorem Ipsum

\subsection{Mask aggregation protocol}
Lorem Ipsum

\subsection{Aggregation and Subtraction protocol}
Lorem Ipsum

\subsection{Pull-based gossip protocol}
Lorem Ipsum

\section{ONet mixed protocol}
Lorem Ipsum

%%%%%%%%%%%%%%%%%%%%
\chapter{Evaluation and Results}
%%%%%%%%%%%%%%%%%%%%

In the evaluation you convince the reader that your design works as intended.
Describe the evaluation setup, the designed experiments, and how the
experiments showcase the individual points you want to prove.

Tentative size: 2 pages.

\section{Parameters used in simulations}
Lorem Ipsum

\section{Comparison of new gossip protocols and BLS CoSi}
Lorem Ipsum

\section{Comparison of mixed gossip in ONet and BLS CoSi}
Lorem Ipsum

%%%%%%%%%%%%%%%%%%%%%%
\chapter{Analysis}
%%%%%%%%%%%%%%%%%%%%%%

The related work section covers closely related work. Here you can highlight
the related work, how it solved the problem, and why it solved a different
problem. Do not play down the importance of related work, all of these
systems have been published and evaluated! Say what is different and how
you overcome some of the weaknesses of related work by discussing the 
trade-offs. Stay positive!

Tentative size: 2 pages.

\section{Limitations}
Lorem Ipsum

\section{Future Work}
Lorem Ipsum

%%%%%%%%%%%%%%%%%%%%
\chapter{Conclusion}
%%%%%%%%%%%%%%%%%%%%

In the conclusion you repeat the main result and finalize the discussion of
your project. Mention the core results and why as well as how your system
advances the status quo.

Tentative size: 1 page.

\cleardoublepage
\phantomsection
\addcontentsline{toc}{chapter}{Bibliography}
\printbibliography

% Appendices are optional
\appendix
%%%%%%%%%%%%%%%%%%%%%%%%%%%%%%%%%%%%%%
\chapter{Appendix: Install and run instructions}
%%%%%%%%%%%%%%%%%%%%%%%%%%%%%%%%%%%%%%

To install do this and this.

To run do this and this.

Tentative size: 1 page.

%%%%%%%%%%%%%%%%%%%%%%%%%%%%%%%%%%%%%%
\chapter{Appendix: Full simulation results}
%%%%%%%%%%%%%%%%%%%%%%%%%%%%%%%%%%%%%%

The simulation results that were not used in the main body of the report, this
will contain figures/charts generated using the simulation results.

Tentative size: 10 pages.



% You need the following items:
% \begin{itemize}
% \item A box
% \item Crayons
% \item A self-aware 5-year old
% \end{itemize}

\end{document}